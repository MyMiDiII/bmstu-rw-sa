\chapter{Классификация существующих решений}

\section{Иерархия методов}

На основе приведенных в предыдущем разделе описаний можно составить иерархию
методов анализа тональности естественно-языковых текстов на основе применяемых
подходов. Данная иерархия представлена на рисунке \ref{img:03}.

\img{10.3cm}{hierarchy}{Иерархия методов анализа тональности текстов}{03}

\section{Сравнение и оценка методов}

Методы анализа тональности естественно-языковых текстов можно оценить по
следующим критериям:
\begin{itemize}
    \item точность;
    \item масштабируемость/зависимость от данных;
    \item сложность реализации.
\end{itemize}
~\\
~\\

\noindent
\captionsetup{format=hang,justification=raggedright,
              singlelinecheck=off,width=14.6cm}
\begin{longtable}[c]{|p{5cm}|p{4cm}|p{3cm}|p{3cm}|}
\caption{Сравнение методов анализа тональности\label{tbl:time_mes}}\\ \hline
    \textbf{Метод} & \textbf{Точность}
                   & \textbf{Критерий 2} & \textbf{Критерий 3}\\ \hline
    На основе правил    & 75\%\cite{article14}    &    & \\ \hline
    На основе словарей  & 67\%\cite{article14}     &    & \\ \hline
    На основе корпусов  &    &    & \\ \hline
    Наивный
    байесовский
    классификатор       &75.5\%\cite{article05}80\%-92\%\cite{article14}    &    & \\ \hline
    Логическая
    регрессия           &93.4\%\cite{article05}    &    & \\ \hline
    Максимум энтропии   &72.6\%\cite{article05}95\%\cite{article14}    &    & \\ \hline
    k-ближайших соседей &    &    & \\ \hline
    Дерево решений      &    &    & \\ \hline
    Случайный лес       &88.39\%\cite{article05}    &    & \\ \hline
    Опорные векторы     &91.15\%\cite{article05}78\%-95\%\cite{article14}  &    & \\ \hline
    Нейронные сети      &87\%\cite{article05}    &    & \\ \hline
\end{longtable}

\section{Вывод}
