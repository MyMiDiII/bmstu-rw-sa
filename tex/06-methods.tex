\chapter{Описание существующих решений}

\section{Лингвистический подход}

Методы использующие лингвистический подход можно разделить на три основные
категории:
\begin{itemize}
    \item методы на основе правил;
    \item методы на основе словарей;
    \item методы на основе корпусов.
\end{itemize}

\subsection{Методы, основанные на правилах}

Работа \textbf{методов на основе правил} реализуется с помощью большого набора
созданных в ручную правил конструкции "если $\rightarrow$ то" \cite{article14}.

Данные алгоритмы имеют отличную производительность в узких
областях тем текстов, однако их обобщение на более широкий круг тем
затруднительно. Также процесс создания необходимых правил является
трудоемким за счет их определения человеком, а не компьютером \cite{article15}.

В целях ускорения процесса разработки для создания набора правил может
использоваться машинное обучение, поэтому в некоторых научных работах
\cite{article16} \cite{article17} данные методы относят к методам машинного
обучения.

\subsection{Методы, основанные на тональных словарях}

\subsection{Методы, основанные на корпусах}

\section{Методы машинного обучения}

\subsection{Наивный Байес}

\subsection{Логическая регрессия}

\subsection{k-ближайших соседей}

\subsection{что-то там про лес и деревья ;)}

\subsection{Нейронные сети}

\section{Гибридные}

Общее описание
