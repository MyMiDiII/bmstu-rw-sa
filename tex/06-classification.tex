\chapter{Классификация существующих решений}

\section{Иерархия методов}

На основе приведенных в предыдущем разделе описаний можно составить иерархию
методов анализа тональности естественно-языковых текстов на основе применяемых
подходов. Данная иерархия представлена на рисунке \ref{img:03}.

\img{10.3cm}{hierarchy}{Иерархия методов анализа тональности текстов}{03}

\section{Сравнение и оценка методов}

Анализ тональности текстов является задачей классификации, для которой
традиционно используют такие метрики эффективности, как точность
(precision), полнота (recall) и F-мера. Вычисление данных
характеристик происходит на основе таблицы сопряженности (таблица
\ref{tab:01}), которая составляется путем сравнения решения системы
относительно класса текста с решением экспертов, которые
формируют тестовые данные~\cite{article18}. 
~\\

\noindent
\begin{table}
\captionsetup{format=hang,justification=raggedright,
              singlelinecheck=off,width=13.8cm}
\begin{center}
\caption{Таблица сопряженности\label{tab:01}}
\begin{tabular}{|c|c|c|c|}
    \hline
    \multicolumn{2}{|c|}{\multirow{2}{*}{}} & \multicolumn{2}{c|}{Оценка
                                                                  эксперта} \\
    \cline{3-4}
    \multicolumn{2}{|c|}{} & Положительная & Отрицательная \\
    \hline
    \multirow{2}{*}{\parbox[c]{2cm}{\centeringОценка системы}} 
    & Положительная & TP & FP \\
    \cline{2-4}
    & Отрицательная & FN & TN \\
    \hline
\end{tabular}
\end{center}
\end{table}

В таблице \ref{tab:01} используются следующие обозначения:
\begin{itemize}
    \item TP (True Positive) --- количество текстов, которые являются
        положительными и которые система определила, как положительные;
    \item FP (False Positive) --- количество текстов, которые являются
        отрицательными и которые система определила, как положительные;
    \item TN (True Negative) --- количество текстов, которые являются
        отрицательными и которые система определила, как отрицательные;
    \item FN (False Negative) --- количество текстов, которые являются
        отрицательными и которые система определила, как положительные.
\end{itemize}

Эти же обозначения используются далее в формулах при определении метрик
эффективности.

\textit{Точность} --- доля текстов, которые действительно принадлежат данному
классу, относительно текстов, которые классификатор причислил к данному
классу (формула \ref{eq:12}).

\begin{equation}\label{eq:12}
    precision = \frac{TP}{TP + FP}
\end{equation}

\textit{Полнота} --- доля текстов, причисленных классификатором к
данному классу, относительно всех текстов, принадлежащий ему в тестовой
выборке~\ref{eq:13}.

\begin{equation}\label{eq:13}
    recall = \frac{TP}{TP + FN}
\end{equation}

\textit{F-мера} --- среднее гармоническое точности и полноты, вычисляющееся
по формуле~\ref{eq:14}.

\begin{equation}\label{eq:14}
    F = \frac{2 \cdot precision \cdot recall}{recision + recall},
\end{equation}

где $F$ --- F-мера.

\noindent
\captionsetup{format=hang,justification=raggedright,
              singlelinecheck=off,width=14.6cm}
\begin{longtable}[c]{|p{5cm}|p{2cm}|p{2cm}|p{1.6cm}|}
\caption{Сравнение методов анализа тональности\label{tab:02}}\\
    \hline
    \textbf{Метод} & \textbf{Точность} & \textbf{Полнота}
                   & \textbf{F-мера}\\
    \hline
    На основе правил\cite{article10}    & 74.2\%
                        & 73.6\%   &  73.8\%\\
    \hline
    На основе словарей \cite{article21}  & 88.0\%
                        & 97.6\%    & 88.9\%\\
    \hline
    На основе корпусов (check)  & 43.0\%
                        & 93.0\% & 55.0\%\\
    \hline
    Наивный
    байесовский
    классификатор       & 89.0\%
                        & 86.0\% & 87.0\%\\
    \hline
    Логистическая
    регрессия           & 88.0\%
                        & 80.0\% & 83.0\%\\
    \hline
    Максимум энтропии   & 65.0\%
                        & 98.0\% & 78.0\%\\
    \hline
    k-ближайших соседей & 73.0\%
                        & 73.0\% & 73.0\%\\
    \hline
    Дерево решений      & 87.0\%
                        & 87.0\% & 87.0\%\\
    \hline
    Случайный лес       & 86.0\%
                        & 93.0\% & 90.0\%\\
    \hline
    Опорные векторы     & 93.0\%
                        & 96.0\% & 94.0\%\\
    \hline
    Нейронные сети      & 75.0\%
                        & 74.0\% & 75.0\%\\
    \hline
\end{longtable}

\section{Вывод}
