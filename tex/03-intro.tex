\chapter*{ВВЕДЕНИЕ}
\addcontentsline{toc}{chapter}{ВВЕДЕНИЕ}

Современное развитие сети Интернет позволяет пользователям ежедневно создавать и
выкладывать в открытый доступ различную информацию, в связи с чем происходит
накопление большого числа данных, одной из наиболее распространенных форм
хранения которых являются тексты на естественном языке. Необходимость анализа
накопленных массивов текстовых данных привела к развитию направления обработки
естественного языка (NLP~---~Natural Language Processing), одной из основных
задач которого стал анализ тональности или сентимент-анализ, заключающийся в
выделении из текстов субъективных мнений и эмоций. Востребованность анализа
тональности во многих областях и невозможность ручной обработки большого числа
текстов послужили разработке и развитию многочисленных автоматических методов,
решающих задачу сентимент-анализа~\cite{article05}.

Целью данной работы является классификация методов анализа тональности
естественно-языковых текстов.

Для достижения поставленной цели решаются следующие задачи:
\begin{itemize}
    \item рассматриваются основные подходы к анализу тональности;
    \item описываются методы анализа тональности, относящиеся к каждому из подходов;
    \item предлагаются критерии оценки качества описанных методов;
    \item сравниваются методы по предложенным критериям оценки;
    \item выделяются методы, показывающие лучшие результаты по одному или
        нескольким критериям.
\end{itemize}
