\chapter{Классификация существующих решений}

\section{Иерархия методов}

На основе приведенных в предыдущем разделе описаний можно составить иерархию
методов анализа тональности естественно-языковых текстов на основе применяемых
подходов. Данная иерархия представлена на рисунке \ref{img:03}.

\img{10.3cm}{hierarchy}{Иерархия методов анализа тональности текстов}{03}

\section{Сравнение и оценка методов}

Анализ тональности текстов является задачей классификации, для которой
традиционно используют такие метрики эффективности, как точность
(precision), полнота (recall) и F-мера. Вычисление данных
характеристик происходит на основе таблицы сопряженности (таблица
\ref{tab:01}), которая составляется путем сравнения решения системы
относительно класса текста с решением экспертов, которые
формируют тестовые данные~\cite{article18}. 
~\\

\noindent
\begin{table}
\captionsetup{format=hang,justification=raggedright,
              singlelinecheck=off,width=13.8cm}
\begin{center}
\caption{Таблица сопряженности\label{tab:01}}
\begin{tabular}{|c|c|c|c|}
    \hline
    \multicolumn{2}{|c|}{\multirow{2}{*}{}} & \multicolumn{2}{c|}{Оценка
                                                                  эксперта} \\
    \cline{3-4}
    \multicolumn{2}{|c|}{} & Положительная & Отрицательная \\
    \hline
    \multirow{2}{*}{\parbox[c]{2cm}{\centeringОценка системы}} 
    & Положительная & TP & FP \\
    \cline{2-4}
    & Отрицательная & FN & TN \\
    \hline
\end{tabular}
\end{center}
\end{table}

В таблице \ref{tab:01} используются следующие обозначения:
\begin{itemize}
    \item TP (True Positive) --- количество текстов, которые являются
        положительными и которые система определила, как положительные;
    \item FP (False Positive) --- количество текстов, которые являются
        отрицательными и которые система определила, как положительные;
    \item TN (True Negative) --- количество текстов, которые являются
        отрицательными и которые система определила, как отрицательные;
    \item FN (False Negative) --- количество текстов, которые являются
        отрицательными и которые система определила, как положительные.
\end{itemize}

Эти же обозначения используются далее в формулах при определении метрик
эффективности.

\textit{Точность} --- доля текстов, которые действительно принадлежат данному
классу, относительно текстов, которые классификатор причислил к данному
классу (формула \ref{eq:12}).

\begin{equation}\label{eq:12}
    precision = \frac{TP}{TP + FP}
\end{equation}

\textit{Полнота} --- доля текстов, причисленных классификатором к
данному классу, относительно всех текстов, принадлежащий ему в тестовой
выборке~\ref{eq:13}.

\begin{equation}\label{eq:13}
    recall = \frac{TP}{TP + FN}
\end{equation}

\textit{F-мера} --- среднее гармоническое точности и полноты, вычисляющееся
по формуле~\ref{eq:14}.

\begin{equation}\label{eq:14}
    F = \frac{2 \cdot precision \cdot recall}{recision + recall},
\end{equation}

где $F$ --- F-мера.

\noindent
\captionsetup{format=hang,justification=raggedright,
              singlelinecheck=off,width=14.6cm}
\begin{longtable}[c]{|p{5cm}|p{2cm}|p{2cm}|p{1.6cm}|}
\caption{Сравнение методов анализа тональности (тосность, полнота,
         F-мера)\label{tab:02}}\\
    \hline
    \textbf{Метод} & \textbf{Точность} & \textbf{Полнота}
                   & \textbf{F-мера}\\
    \hline
    На основе правил    & 74.23\% 
                        & 73.67\% & 73.94\%\\
    \hline
    На основе словарей  & 97.56\% 
                        & 88.89\% & 93.02\%\\
    \hline
    На основе корпусов  & 43.10\%
                        & 94.34\% & 59.17\%\\
    \hline
    Наивный
    байесовский
    классификатор       & 75.49\%
                        & 52.44\% & 67.70\%\\
    \hline
    Логистическая
    регрессия           & 76.95\% 
                        & 80.13\% & 78.51\%\\
    \hline
    Максимум энтропии   & 65.21\%
                        & 98.04\% & 78.32\%\\
    \hline
    k-ближайших соседей & 59.14\% 
                        & 96.14\% & 73.24\%\\
    \hline
    Дерево решений      & 96.67\% 
                        & 64.52\% & 77.38\%\\
    \hline
    Случайный лес       & 89.47\% 
                        & 76.53\% & 82.49\%\\
    \hline
    Опорные векторы     & 83.87\%
                        & 83.89\% & 83.88\%\\
    \hline
    Нейронные сети      & 85.30\% 
                        & 88.41\% & 86.83\%\\
    \hline
\end{longtable}

ЛК-1 --- особенности подготовки обучающих данных 

ЛК-2 --- необходимость повторного обучения при изменении тестовых данных

\noindent
\captionsetup{format=hang,justification=raggedright,
              singlelinecheck=off,width=14.6cm}
\begin{longtable}[c]{|p{5cm}|p{5cm}|p{3cm}|}
\caption{Сравнение методов анализа тональности (обучающие данные)\label{tab:03}}\\
    \hline
    \textbf{Метод} & \textbf{ЛК-1} & \textbf{ЛК-2}\\
    \hline
    На основе правил    & ручная
                          (составление правил) & требуется\\
    \hline
    На основе словарей  & смешанная
                          (ручной выбор базовых слов,
                           автоматический поиск в словаре) & требуется\\
    \hline
    На основе корпусов  & смешанная
                          (ручной выбор базовых слов,
                           автоматический поиск в корпусе текстов & требуется\\
    \hline
    Наивный
    байесовский
    классификатор       & автоматическая
                          расчет вероятностей
                          принадлежности слов классам  & не требуется\\
    \hline
    Логистическая
    регрессия           & автоматическая
                          расчет коэффициентов
                          логистической функции & \\
    \hline
    Максимум энтропии   & автоматическая
                          расчет вероятностей
                          принадлежности слов классам & \\
    \hline
    k-ближайших соседей & автоматическая
                          нормализация & \\
    \hline
    Дерево решений      & автоматическая
                          построение дерева & требуется\\
    \hline
    Случайный лес       & автоматическая
                          (построение множества деревьев) & не требуется\\
    \hline
    Опорные векторы     & автоматическая & не требуется\\
    \hline
    Нейронные сети      & смешанная
                          (настройка параметров сети) & не требуется \\
    \hline
\end{longtable}

\section{Вывод}
