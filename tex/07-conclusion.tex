\chapter*{ЗАКЛЮЧЕНИЕ}
\addcontentsline{toc}{chapter}{ЗАКЛЮЧЕНИЕ}

В рамках научно-исследовательской работы была проведена классификация методов
анализа тональности естественно-языковых текстов.

В результате сравнения методов были выделены: метод на основе тональных
словарей, как наиболее точный из представленных, а также нейронные сети и метод
опорных векторов, как самые точные из методов, не требующих ручной подготовки
данных и повторной настройки при расширении предметной области анализируемых
текстов.

В итоге, в ходе данной работы:
\begin{itemize}
    \item рассмотрены основные подходы к анализу тональности;
    \item описаны методы аналаза тональности естественно-языковых текстов;
    \item предложены критерии оценки, по которым проведено сравнение
        описанных методов;
    \item выделены методы, показывающие лучшие результаты по предложенным
        критериям.
\end{itemize}

Таким образом, все поставленные задачи были выполнены, а цель достигнута.
