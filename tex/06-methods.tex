\chapter{Описание существующих решений}

\section{Лингвистический подход}

Методы использующие лингвистический подход можно разделить на три основные
категории:
\begin{itemize}
    \item методы на основе правил;
    \item методы на основе словарей;
    \item методы на основе корпусов.
\end{itemize}

\subsection{Методы, основанные на правилах}

Работа \textbf{методов на основе правил} реализуется с помощью большого набора
созданных в ручную правил конструкции "если $\rightarrow$ то" \cite{article14}.

Данные алгоритмы имеют отличную производительность в узких
областях тем текстов, однако их обобщение на более широкий круг тем
затруднительно. Также процесс создания необходимых правил является
трудоемким за счет их определения человеком, а не компьютером \cite{article15}.

В целях ускорения процесса разработки для создания набора правил может
использоваться машинное обучение, поэтому в некоторых научных работах
\cite{article16} \cite{article17} данные методы относят к методам машинного
обучения.

\subsection{Методы, основанные на тональных словарях}

Первый лингвистический метод основан на тональных словарях. Тональный словарь
представляет собой набор слов или биграмм, которым задается определенный вес
принадлежности к позитивному или негативному классу. При анализе текста каждое
слово ищется в этом словаре, и его вес записывается. Если слова нет в словаре,
то его класс считается нейтральным, и вес равняется нулю. После того как все
веса получены, высчитывается принадлежность данного текста к определенному
классу тональности \cite{article14}.

Данный подход основан на использовании словарей с заранее подготовленными
вручную шаблонами эмоционально важных слов и словосочетаний с их эмоциональными
оценками. При использовании данного подхода в тексте ищутся пересечения со
словарем. Затем по сумме оценок найденных пересечений определяется тональность
заданного текста. Данный подход показывает хорошие результаты для некоторых
областей. Основной недостаток данного подхода в большой сложности подготовки
словарей, надо хорошо знать предметную область, для которой составляется
словарь. Второй недостаток — это плохая масштабируемость, нельзя использовать
один и тот же словарь для разных предметных областей. Одинаковые термины в
различных областях могут вносить разный вес в степень эмоциональной окраски
\cite{article9}

Подход на основе словаря. При словарном подходе некоторые слова выбираются в
качестве начального слова, и эти слова используются для поиска синонимов, чтобы
увеличить размер набора слов. Для увеличения размера используются
онлайн-словари. Исходные слова - это слова мнения, которые являются уникальными
и важными в корпусе \cite{article16}.

В этом подходе, прежде всего, вручную собирается небольшой набор слов
настроения, которые известны как "seed words", с их известной положительной или
отрицательной ориентацией. Затем этот набор увеличивается путем поиска их
синонимов и антонимов в WordNet или другом онлайн-словаре.  Новые слова
добавляются к существующему списку. Затем запускается следующая итерация.
Итерация должна быть остановлена, если не найдено ни одного нового слова.
Наконец, для очистки списка используется набор ручной проверки \cite{article4}.

\subsection{Методы, основанные на корпусах}

Корпус - это, по сути, термин, который является кластером письменных текстов,
как группа некоторых письменных текстов, часто по очень точному вопросу. В этом
случае пользователи используют корпус текстов для составления списка семян,
который находится в организованной ситуации \cite{article18}.

Подход, основанный на корпусе, начинает с исходного списка слов, выражающих
мнение, а затем находит другие слова, выражающие мнение, в большом корпусе,
чтобы помочь найти слова, выражающие мнение, с контекстно-специфической
ориентации \cite{article2}.

\section{Методы машинного обучения}

\subsection{Наивный байесовский классификатор}

\subsection{Логическая регрессия}

\subsection{Метод максимума энтропии}

\subsection{k-ближайших соседей}

\subsection{Деревья решений}

\subsection{Случайный лес}

\subsection{Метод опорных векторов}

\subsection{Нейронные сети}

\section{Гибридные}

Общее описание
