\chapter{Классификация существующих решений}

В данном разделе приводится иерархия существующих решений, предлагаются
критерии оценки методов и проводится сравнение по выделенным критериям.

\section{Иерархия методов}

На основе приведенных в предыдущем разделе описаний можно составить иерархию
методов анализа тональности естественно-языковых текстов на основе применяемых
подходов. Данная иерархия представлена на рисунке \ref{img:03}.

\img{10.3cm}{hierarchy}{Иерархия методов анализа тональности текстов}{03}

\section{Сравнение и оценка методов}

Анализ тональности текстов является задачей классификации, для которой
традиционно используют такие метрики эффективности, как точность
(precision), полнота (recall) и F-мера. Вычисление данных
характеристик происходит на основе таблицы сопряженности (таблица
\ref{tab:01}), которая составляется путем сравнения решения системы
относительно класса текста с решением экспертов, которые
формируют тестовые данные~\cite{article18}. Далее приведенные выше величины
называются метриками точности.

\noindent
\captionsetup{format=hang,justification=raggedright,
              singlelinecheck=off,width=13.7cm}
\begin{longtable}[c]{|c|c|c|c|}
\caption{Таблица сопряженности\label{tab:01}}\\
    \hline
    \multicolumn{2}{|c|}{\multirow{2}{*}{}} & \multicolumn{2}{c|}{Оценка
                                                                  эксперта} \\
    \cline{3-4}
    \multicolumn{2}{|c|}{} & Положительная & Отрицательная \\
    \hline
    \multirow{2}{*}{\parbox[c]{2cm}{\centeringОценка системы}} 
    & Положительная & TP & FP \\
    \cline{2-4}
    & Отрицательная & FN & TN \\
    \hline
\end{longtable}

В таблице \ref{tab:01} используются следующие обозначения:
\begin{itemize}
    \item TP (True Positive) --- количество текстов, которые являются
        положительными и которые система определила, как положительные;
    \item FP (False Positive) --- количество текстов, которые являются
        отрицательными и которые система определила, как положительные;
    \item TN (True Negative) --- количество текстов, которые являются
        отрицательными и которые система определила, как отрицательные;
    \item FN (False Negative) --- количество текстов, которые являются
        отрицательными и которые система определила, как положительные.
\end{itemize}

Эти же обозначения используются далее в формулах при определении метрик
эффективности.

\textit{Точность (precision)} --- доля текстов, которые действительно принадлежат данному
классу, относительно текстов, которые классификатор причислил к данному
классу. Вычисляется по формуле~(\ref{eq:12}).

\begin{equation}\label{eq:12}
    precision = \frac{TP}{TP + FP}
\end{equation}

\textit{Полнота (recall)} --- доля текстов, причисленных классификатором к
данному классу, относительно всех текстов, принадлежащий ему в тестовой
выборке. Вычисляется по формуле~(\ref{eq:13}).

\begin{equation}\label{eq:13}
    recall = \frac{TP}{TP + FN}
\end{equation}

\textit{F-мера} --- среднее гармоническое точности и полноты, вычисляющееся
по формуле~(\ref{eq:14}).

\begin{equation}\label{eq:14}
    F = \frac{2 \cdot precision \cdot recall}{recision + recall},
\end{equation}

где $F$ --- F-мера.

Также методы анализа тональности основываются на заранее подготовленных данных
(набора слов, правил, коэффициентов~и~т.~п.), с помощью которых происходит
классификация. Такая подготовка требует дополнительных затрат и осуществляется
различными путями, поэтому в качестве одного из критериев выбран способ
проведения подготовки данных:
\begin{itemize}
    \item ручной (данные подготавливаются людьми);
    \item автоматический (подготовка осуществляется с помощью вычислительной
        техники);
    \item смешанный (сначала осуществляется ручная подготовка данных, на основе
        которой проводится дополнительная автоматическая подготовка).
\end{itemize}

Требуемая во всех классификаторах предварительная разметка данных не
учитывается.

При анализе тональности также может возникуть необходимость расширения
предметной области текстов, классификация которых происходит, для чего
может постребоваться повторная настройка или обучение классификатора.
Необходимость такой настройки в данной работе рассматривается как критерий
оценки методов.

Результаты сравнения методов анилиза тональности приведены в
таблице~\ref{tab:02}. Для краткости записи в данной таблице используются
следующие обозначения описанных критериев:
\begin{itemize}
    \item К1 --- точность;
    \item К2 --- полнота;
    \item К3 --- F-мера;
    \item K4 --- способ проведения подготовки данных;
    \item К5 --- необходимость повторной настройки классификатора при расширении
        предметной области анализируемых текстов.
\end{itemize}

\clearpage
\noindent
\captionsetup{format=hang,justification=raggedright,
              singlelinecheck=off,width=16.8cm}
\begin{longtable}[Hc]{|p{3.2cm}|p{1.5cm}|p{1.5cm}|p{1.5cm}|p{3.5cm}|p{3cm}|}
\caption{Сравнение методов анализа тональности\label{tab:02}}\\
    \hline
    \multicolumn{1}{|c}{\textbf{Метод}} & \multicolumn{1}{|c|}{\textbf{K1}} &
    \multicolumn{1}{c|}{\textbf{K2}} & \multicolumn{1}{c}{\textbf{K3}} &
    \multicolumn{1}{|c|}{\textbf{К4}} & \multicolumn{1}{c|}{\textbf{K5}}\\
    \hline
    \mbox{На~основе}
    \linebreakправил~\cite{article18}    & 74.23\% & 73.67\% & 73.94\%
                                     & ручной & требуется \\*
    \hline
    \mbox{На~основе}
    \linebreakсловарей~\cite{article18}  & 97.56\% & 88.89\% & 93.02\%
                                     & смешанный & требуется\\*
    \hline
    \mbox{На~основе}
    \linebreakкорпусов~\cite{article18}  & 43.10\% & 94.34\% & 59.17\%
                                     & смешанный & требуется\\*
    \hline
    Наивный\linebreak
    Байес~\cite{article18}               & 75.49\% & 52.44\% & 67.70\%
                                     & автоматический & не требуется\\*
    \hline
    Логистическая
    регрессия ~\cite{article18}          & 76.95\% & 80.13\% & 78.51\%
                                     & автоматический & не требуется\\
    \hline
    Максимум\linebreak
    энтропии~\cite{article18}            & 65.21\% & 98.04\% & 78.32\%
                                     & автоматический & требуется\\
    \hline
    k-ближайших
    \linebreakсоседей~\cite{article18}   & 59.14\% & 96.14\% & 73.24\%
                                     & автоматический & не требуется\\
    \hline
    Дерево\linebreak
    решений~\cite{article18}             & 96.67\% & 64.52\% & 77.38\%
                                     & автоматический & требуется\\
    \hline
    Случайный лес~\cite{article18}       & 89.47\% & 76.53\% & 82.49\%
                                     & автоматический & не требуется\\
    \hline
    Опорные\linebreak
    векторы~\cite{article18}             & 83.87\% & 83.89\% & 83.88\%
                                     & автоматический & не требуется\\
    \hline
    Нейронные\linebreak
    сети~\cite{article18}                & 85.30\% & 88.41\% & 86.83\%
                                     & смешанный & не требуется \\
    \hline
\end{longtable}

\section{Вывод}

Таким образом, по данным таблицы \ref{tab:02} наиболее точным методом является
метод на основе словарей, однако в нем подготовка данных осуществляется с
участием человека, а расширение предметных областей текстов требует повторной
настройки классификатора, как и в случае других методов лингвистического
подхода, показывающих метрики точности, близкие к метрикам методов машинного обучения. Наиболее точным методом с автоматической
подготовкой данных для классификации и c отсутствием необходимости повторной
настройки является метод опорных векторов.  Для повышения точности классификации
можно использовать нейронные сети, однако придется пожертвовать временем на их
настройку. Остальные методы показывают приемлемые метрики точности в районе
60-80 процентов, что говорит об их применимости к решению конкретных задач для
определенных предметных областей текстов.


