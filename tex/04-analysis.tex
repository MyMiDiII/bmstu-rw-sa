\chapter{Анализ предметной области}

В данном разделе обоснована актуальность задачи, представлены основные
определения и формализация задачи.

\section{Актуальность задачи}

В современном мире огромную роль в жизни каждого человека играет Интернет.
Люди общаются в социальных сетях, ведут блоги, оставляют отзывы о товарах,
услугах, фильмах, книгах и т.~п. За счет этого в открытом доступе находится
огромный объем данных, который позволяет проводить точные анализы для
решения каких-либо задач.  

Большая часть накопленных данных представлена в виде текстовой информации,
поэтому становится актуальной задача анализа текстов на естественном
языке~\cite{article01}. Одной из этих задач является анализ тональности или
сентимент-анализ. За счет того, что такой анализ может быть проведен для текста,
написанного на любую тему, его применение возможно во многих сферах:
\begin{itemize}
    \item мониторинг общественного мнения относительно товаров
        и услуг, в том числе в режиме реального времени, с целью определения их
        достоинств и недостатков с точки зрения покупателей и улучшения их
        характеристик~\cite{article02};
    \item анализ политических и социальных взглядов пользователей (например,
        влияние мер, предпринятых для борьбы с вирусом COVID-19, на жизнь
        людей);
    \item исследование рынка и прогнозирование цен на акции~\cite{article03};
    \item выявление случаев эмоционального насилия и пресечение
        противоправных действий~\cite{article04}.
\end{itemize}

Решение описанных задач требует анализа большого количества текстов, что делает
невозможной их ручную обработку. Также при оценке тональности текста человеком
трудно соблюсти критерии этой оценки~\cite{article02}. Таким образом, возникает
необходимость в автоматизированных системах анализа.

При этом в отличие от традиционной обработки текста в анализе тональности
незначительные вариации между двумя элементами текста существенно меняют смысл
(например, добавление частицы <<не>>). Обработку естественного языка затрудняет
обильное использования носителями средств выразительности и переносных значений
слов и фраз. Также основной из проблем сентимент-анализа является разная окраска
одного и того же слова в текстах на различные тематики: слово, которое считается
положительным в одной, в то же время считается отрицательным в
другой~\cite{article03}.

С учетом широкого применения анализ тональности и описанных сложностей,
возникает необходимость в формализации поставленной проблемы и разработки
методов для ее решения.

\section{Основные определения}

\textit{Анализ тональности текста} (или \textit{сентимент-анализ}) --- область
компьютерной лингвистики, ориентированная на извлечение из текстов субъективных
мнений и эмоций. \textit{Тональность} --- это мнение, отношение и эмоции автора
по отношению к объекту, о котором говорится в тексте. Чаще всего под задачей
анализа тональности текста понимают определение текста к одному из двух классов:
<<положительный>> или <<отрицательный>>. В некоторых случаях добавляют третий
класс <<нейтральных>> текстов~\cite{article05}.

В настоящее время выделяют три основных подхода к определению тональности
текста~\cite{article05}:
\begin{itemize}
    \item \textit{лингвистический подход} предполагает анализ лексики в тексте на
        основе заранее созданных словарей, правил и шаблонов;

    \item \textit{подход, основанный на машинном обучении}, строится на обучении и
        автоматическом построении классифицирующей функции на основе
        некоторых данных, полученных из текстов, тональность которых
        известна;

    \item \textit{гибридный подход} сочетает в себе подходы как на основе
        словарей, правил и шаблонов, так и на основе машинного обучения;
\end{itemize}


Несмотря на различные подходы, можно выделить несколько этапов анализа
тональности текста~\cite{article09}:
\begin{itemize}
    \item предварительная обработка текста;
    \item извлечение информативных признаков или векторизация текста;
    \item классификация;
    \item оценка результата работы.
\end{itemize}

Первым этапом анализа тональности является предварительная обработка текста,
которая также происходит в несколько этапов:
\begin{itemize}
    \item приведение текста к \textit{единому регистру} для сокращения
        количества слов, которые необходимо хранить одновременно;
    \item \textit{удаление пунктуации и шума} (упоминаний пользователей, ссылок,
        хештегов), \textit{исправление опечаток и ошибок}~\cite{article06};
    \item \textit{токенизация} или разбиение исходного текста на лексемы, в простейшем
        случае --- разбиение по пробельным символам~\cite{article07};
    \item \textit{удаление стоп-слов}, то есть слов не несущих никакой смысловой
        нагрузки, с целью повышения точности;
    \item \textit{стемминг} или \textit{лемматизация} --- приемы приведения слов
        форм слова к общему виду; в случае стемминга происходит получение корня
        слова путем отбрасывания приставок, суффиксов и окончаний, в случае
        лемматизации --- воспроизводится начальная форма слова, то есть та форма,
        которая представлена в словаре~\cite{article06};
    \item обработка отрицаний~\cite{article08}.
\end{itemize}

На втором этапе анализа происходит представление текста в виде вектора. Для
этого используется один из многочисленных методов, таких как <<мешок слов>> (Bag
of Words, представляет текст в виде неупорядоченного набора слов), Word2Vec
(нейронная сеть, генерирующая векторы слов)~и~др. Этап векторизации текста для
методов лингвистического подхода не является обязательным, так как в данном
случае происходит работа непосредственно с текстами, а не с их
векторами~\cite{article09}.

На третьем этапе происходит собственно определение тональности текста с
использованием одного из методов, описанных ниже.

На последнем этапе производится оценка результатов работы классификатора.

\section{Формализация задачи}

В данной работе ставится задача анализа методов определения принадлежности
заданного естественно-языкового текста к одному из двух классов:
\begin{itemize}
    \item положительный;
    \item отрицательный.
\end{itemize}

При этом определяется лишь факт принадлежности тому или иному классу, и оценка
вероятности отношения текста к каждому классу не проводится.

