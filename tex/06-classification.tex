\chapter{Классификация существующих решений}

\section{Иерархия методов}

На основе приведенных в предыдущем разделе описаний можно составить иерархию
методов анализа тональности естественно-языковых текстов на основе применяемых
подходов. Данная иерархия представлена на рисунке \ref{img:03}.

\img{10.3cm}{hierarchy}{Иерархия методов анализа тональности текстов}{03}

\section{Сравнение и оценка методов}

Методы анализа тональности естественно-языковых текстов можно оценить по
следующим критериям:
\begin{itemize}
    \item точность;
    \item критерий 2;
    \item критерий 3.
\end{itemize}
~\\
~\\

\noindent
\captionsetup{format=hang,justification=raggedright,
              singlelinecheck=off,width=13.5cm}
\begin{longtable}[c]{|p{5cm}|p{2cm}|p{3cm}|p{3cm}|}
\caption{Сравнение методов анализа тональности\label{tbl:time_mes}}\\ \hline
    \textbf{Метод} & \textbf{Точность}
                   & \textbf{Критерий 2} & \textbf{Критерий 3}\\ \hline
    На основе правил    &    &    & \\ \hline
    На основе словарей  &    &    & \\ \hline
    На основе корпусов  &    &    & \\ \hline
    Наивный
    байесовский
    классификатор       &    &    & \\ \hline
    Логическая
    регрессия           &    &    & \\ \hline
    Максимум энтропии   &    &    & \\ \hline
    k-ближайших соседей &    &    & \\ \hline
    Дерево решений      &    &    & \\ \hline
    Случайный лес       &    &    & \\ \hline
    Опорные векторы     &    &    & \\ \hline
    Сверточная
    нейронная сеть      &    &    & \\ \hline
    Рекуррентная
    нейронная сеть      &    &    & \\ \hline
\end{longtable}

\section{Вывод}
