\chapter{Анализ предметной области}

\section{Основные определения}

Sentiment analysis (SA) или анализ мнений извлекает и анализирует субъективную
информацию из различных источников, таких как Интернет, социальные сети и другие
источники, чтобы определить мнение людей с помощью обработки естественного языка
(NLP), вычислительной лингвистики и анализа текста.  Эта проанализированная
информация дает представление о чувствах или отношении общества к определенным
предметам, лицам или идеям и определяет контекстуальную полярность информации.
\cite{article2}


Тональность

Анализ тональности

Предобработка

Корпус

Словарь

Позитивный/негативный/нейтральный

\section{Актуальность задачи}

Появление Интернета -> высказываение мнений -> возможность их анализа
мнений/настроений. В этом и есть задача анализа. Здесь про текст. Сферы (продукты,
политика, акции, социальное положение). Популярность -> много данных ->
возможность получения точных результатов.

Но есть проблемы:

Накоплены большие объемы информации -> большая часть -- текстовые данные.
Несмотря на то, что текстовые данные содержат массу полезных знаний, они  по своей природе являются
неструктурированными, а существующие объемы данных непозволяются обработать
тексты в ручную. При этом в отличие от традициаонной обработки текста в
анализе настроений незначительные вариации между двумя элементами текста
существенно меняют смысл (например, добавление частицы "не").
Обработку русского языка затрудняет обильное использования носителями средств
выразительности и переносных значений слов и раз.

Наличие большого количества сфер применений, необходимость решения описанных
проблем приводит к созданию "каких-то там" методов.

\section{Формализация задачи}

Три класса текста. Задача в определении текста к одному из класов (без
вероятности).

\section{Препроцессинг}

единый регистр

удаление пунктуации

лемматизация/стемминг

удаление стоп слов

удаление шума (хэш-тэгов, ссылок и тд)

