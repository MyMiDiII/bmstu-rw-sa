\chapter{Анализ предметной области}

\section{Актуальность задачи}

В современном мире огромную роль в жизни каждого человека играет Интернет.
Люди общаются в социальных сетях, ведут блоги, оставляют отзывы о товарах,
услугах, фильмах, книгах и т.~п. За счет этого в открытом доступе находится
огромный объем данных, который позволяет проводить точные анализы для
решения каких-либо задач.  

Большая часть накопленных данных представлена виде текстовой информации, поэтому
становится актуальной задача анализа текстов на естественном языке.
\cite{article8} Одной из этих задач является анализ тональности и
сентимент-анализ.  За счет того, что такой анализ может быть проведен для
текста, написанного на любую тему, его применение возможно во многих сферах:
\begin{itemize}
    \item мониторинг общественного мнения \cite{article1} относительно товаров и
        услуг, в том числе в режиме реального времени, с целью определения их
        достоинств и недостатков с точки зрения покупателей и улучшения их
        характеристик \cite{article3};
    \item анализ политических и социальных взглядов пользователей (например,
        влияние мер, предпринятых для борьбы с вирусом COVID-19, на жизнь
        людей) \cite{article3};
    \item исследование рынка и прогнозирование цен на акции
        \cite{article3};
    \item выявление случаев эмоционального насилия и пресечение
        противоправных действий \cite{article6}.
\end{itemize}

Решение описанных задач требует анализ большого количества текстов, что
делает невозможным их ручную обработку. Также при оценке тональности текста
человеком трудно соблюсти критерии этой оценки. Таким образом, возникает
необходимость в автоматизированных системах анализа.

При этом в отличие от традиционной обработки текста в анализе тональности
незначительные вариации между двумя элементами текста существенно меняют смысл
(например, добавление частицы "не"). Обработку естественного языка затрудняет
обильное использования носителями средств выразительности и переносных значений
слов и фраз. Также основной из проблем сентимент-анализа является разная окраска
одного и того слова в текстах на различные тематики: слово, которое считается
положительным в одной, в то же время считается отрицаельным в другой.

С учетом широкого применения анализ тональности и описанных сложностей,
возникает необходимость в формализации поставленной проблемы и разработки
методов для её решения.

\section{Основные определения}

\textbf{Анализ тональности текста} (sentiment analysis) -- область компьютерной
лингвистики, ориентированная на извлечение из текстов субъективных мнений и
эмоций. \textbf{Тональность} -- это мнение, отношение и эмоции автора по
отношению к объекту, о котором говорится в тексте.  Чаще всего под задачей
анализа тональности текста понимают определение текста к одному из двух классов:
"положительный"\ или "отрицательный". В некоторых случаях добавляют третий класс
"нейтральных"\ текстов.\cite{article9}

В настоящее время выделяют три основных подхода к определению тональности
текста:
\begin{itemize}
    \item лексический анализ;

        Абзац про него

    \item методы машинного обучения;

        Абзац про него

    \item  гибридные методы;

        Абзац про него
\end{itemize}

Несмотря на различные подходы к решению задачи анализа тональности, во всех
подходах требуется предварительная обработка текста, основыми этапами которой
являются:

единый регистр

удаление пунктуации

лемматизация/стемминг

удаление стоп слов

удаление шума (хэш-тэгов, ссылок и тд)


\section{Формализация задачи}

В данной работе ставится задача анализа методов определения принадлежности
заданного естественно-языкового текста к одному из двух классов:
\begin{itemize}
    \item положительный;
    \item отрицательный.
\end{itemize}

При этом определяется лишь \textbf{факт} принадлежности тому или иному
классу, и оценка вероятности отношения текста к каждому классу не проводится.

Такая задача рассматривается во многих статьях, которые использовались при
написании данной работы, что упрощает задачу сравнения методов ее решения и
способствует получению объективных результатов оценки.

