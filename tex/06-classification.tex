\chapter{Классификация существующих решений}

\section{Иерархия методов}

На основе приведенных в предыдущем разделе описаний можно составить иерархию
методов анализа тональности естественно-языковых текстов на основе применяемых
подходов. Данная иерархия представлена на рисунке \ref{img:03}.

\img{10.3cm}{hierarchy}{Иерархия методов анализа тональности текстов}{03}

\section{Сравнение и оценка методов}

Анализ тональности текстов является задачей классификации, для которой
традиционно используют такие метрики эффективности, как точность
(precision), полнота (recall) и F-мера. Вычисление данных
характеристик происходит на основе таблицы сопряженности (таблица
\ref{tab:01})~\cite{article18}. 

\noindent
\begin{table}
\captionsetup{format=hang,justification=raggedright,
              singlelinecheck=off,width=14.6cm}
\begin{center}
\caption{Таблица сопряженности\label{tab:01}}
\begin{tabular}{|c|c|c|c|}
\hline
\multicolumn{2}{|c|}{\multirow{2}{*}{}} & \multicolumn{2}{c|}{Оценка эксперта} \\
\cline{3-4}
\multicolumn{2}{|c|}{} & Положительная & Отрицательная \\
\hline
\multirow{2}{*}{\parbox[c]{2cm}{\centeringОценка системы}} & Положительная & TP & FP \\
\cline{2-4}
                                & Отрицательная & FN & TN \\
\hline
\end{tabular}
\end{center}
\end{table}


\noindent
\captionsetup{format=hang,justification=raggedright,
              singlelinecheck=off,width=14.6cm}
\begin{longtable}[c]{|p{5cm}|p{4cm}|p{3cm}|p{3cm}|}
\caption{Сравнение методов анализа тональности\label{tab:02}}\\ \hline
    \textbf{Метод} & \textbf{Точность}
                   & \textbf{Критерий 2} & \textbf{Критерий 3}\\ \hline
    На основе правил    & 75\%\cite{article14}    &    & \\ \hline
    На основе словарей  & 67\%\cite{article14}     &    & \\ \hline
    На основе корпусов  &    &    & \\ \hline
    Наивный
    байесовский
    классификатор       &75.5\%\cite{article05}80\%-92\%\cite{article14}    &    & \\ \hline
    Логическая
    регрессия           &93.4\%\cite{article05}    &    & \\ \hline
    Максимум энтропии   &72.6\%\cite{article05}95\%\cite{article14}    &    & \\ \hline
    k-ближайших соседей &    &    & \\ \hline
    Дерево решений      &    &    & \\ \hline
    Случайный лес       &88.39\%\cite{article05}    &    & \\ \hline
    Опорные векторы     &91.15\%\cite{article05}78\%-95\%\cite{article14}  &    & \\ \hline
    Нейронные сети      &87\%\cite{article05}    &    & \\ \hline
\end{longtable}

\section{Вывод}
